\documentclass[12pt]{article}

\usepackage[shortlabels]{enumitem}

% Following lines copied from stackoverlow
\usepackage{color}   %May be necessary if you want to color links
\usepackage{hyperref}
\hypersetup{
    colorlinks=true, %set true if you want colored links
    linktoc=all,     %set to all if you want both sections and subsections linked
    linkcolor=blue,  %choose some color if you want links to stand out
}

\title{UMD Spring 2020:\\Pass/Low Pass/Fail\\Grading System}
\author{Chandan Murthy\thanks{cmurthy [AT] umd [DOT] edu}}
\begin{document}
\maketitle

\tableofcontents

\newpage
\section{Motivation for Writing this Document}
A little bit about me. I am Chandan, a sophomore CS + Math double major. I am also a TA for CMSC330, the Programming Languages class here at UMD (this fact is repeated for emphasis somewhere in this argument). I wrote this document during Spring Break instead of doing my work. I hope you read through my arguments and give it some thought.

\par Before continuing, I want to clarify that all the views and positions taken in this document are mine. At the time of me writing this, I have received at least four messages from different people, containing the same link to a petition for UMD to employ a Double-A grading system. This document was conceived in one sitting, initially as a transcription of my frustrations with these messages (how could anyone thing a Double-A grading system is a good replacement to our current system??). As a result, I did not put \textit{too} much thought into this, so there many very well be grammatical and/or logical mistakes, repetitions, etc. Please feel free to refute any of my positions and let me know why I am wrong. Now let's proceed to the actual argument...

\newpage
\section{Introduction}
\subsection{Abstract}
The Coronavirus pandemic has resulted in the closure of the UMD campus, and as a result all remaining classes for this semester will be held online. As a response to this new and strange situation, many are calling for a \textbf{Pass/Fail} or \textbf{Double-A system} (in which everyone gets either an A or an A-) to replace the current \textbf{A-F system} for grading. The changes in grading are being proposed for this semester only. In this document, I lay out the issues with both of these grading systems, and I propose a new system that I believe will be the best replacement for the system we have now.  
\subsection{The Need for a New Grading System}
There are compelling reasons to switch to a new grading system for the remainder of this semester. Learning material in-person vs. online each comes with its advantages, but forcing all students to take classes online will result in grades that are unfair or unrepresentative of a student's mastery of the material in the courses they are taken. For example, online classes will be hard for teachers to adapt to and will be less engaging for all students. Personally, as a Math major, I will have to start learning advanced mathematical concepts and proofs from textbooks and my teacher's notes, instead of effectively learning the material in real time from a blackboard. Office hours and discussion sections for all classes will be held online, which would further hinder engagement and learning. Furthermore, for many students, including those from different states and countries, attending live lectures will simply not be an option. Many students will also have trouble completing work due to their living situation, familial responsibilities, household distractions, inconsistent internet access, etc. 
Through no fault of their own, most students, under the current grading system, will not be able to work as hard as they can, and thus cannot achieve their desired grades. Therefore, we should change the grading system for all students (for this semester) to reflect the many barriers imposed by this unusual situation we are all going through.

\newpage
\subsection{Pitfalls of a Double-A Grading System}
Personally, I think that the Double-A system is one of the worst solutions for this problem, and I hope it does not get implemented. Why? Consider the following:

\begin{enumerate}
    \item There is no incentive to learn anything or do well in class. This semester would be meaningless. A student doing all the work and getting an A gets a marginal GPA increase compared to someone doing no work at all.
    \item You could put in no work and still get an A-. In what world does this make sense? Even if you cannot attend live lectures, are working, taking care of family, etc. you still do not deserve an A- for doing the minimal amount of work necessary.
    \item If artificially inflates every single student's GPA for no good reason.
    \item It would ignore everyone's progress so far in the semester. I am a TA for CMSC330, which has over 600 students, and which is required for all CS majors. It would certainly not be fair for students in this class if all 600+ of them got an free A- in an upper level benchmark class. My work as a TA would have been all for naught, and they get a free pass to take any 400 level class of their choosing.
    \item Suppose you see no problem with all of the above points. Say we employ this grading system for this semester. What about the notion of a ``prerequisite''? This would affect almost everyone who is not graduating this semester. For example, everyone taking Algorithms or Organic Chemistry or any other ``hard'' or ``benchmark'' class would be able to take upper-level classes, and I guarantee many students would not have the knowledge or skills necessary for a 400 level class. Or consider even someone taking Calc I who has to take Calc II for their major. What if they would have received a C- or a literal failing grade under normal circumstances but instead get an A-? Now they get to take Calc II with in an over-crowded class and everything would be much worse.
    \item Note that an A/A-/Fail system would mitigate some of the above issues, but that system would still be bad.
\end{enumerate}

\subsection{Pitfalls of a Pass/Fail Grading System}
I also dislike a Pass/Fail system.
\begin{enumerate}
    \item It is too restrictive. A binary grading systems ignore the progress made so far in the semester. It doesn't matter if you got As or Cs in all your classes or midterms so far, since you would likely get a final grade of ``Pass'' either way. Maybe the rest of the semester could be Pass/Fail or have lenient grading but the final grade should incorporate the grades received so far.
    \item It would result in everyone's GPAs being static for one semester. I was really hoping to increase my GPA this semester but a Pass/Fail system would not allow me to do that.
\end{enumerate}


\newpage
\section{Main Proposal}

Yale Law School employs an Honors/Pass/Low Pass/Fail grading system. I propose the following alternative to that system, which I contend is better than Pass/Fail, Double-A, and the current A-F grading system.

\vspace{25pt}
\underline{\textbf{Pass/Low Pass/Fail:}}
\begin{itemize} 
\item A grade of 80-100 would be a Pass;
\item A grade of 60-79 would be a Low Pass;
\item A grade of 0-59 would be a Fail. 
\end{itemize}

\subsection{Benefits of a Pass/Low Pass/Fail Grading System}
This is obviously better than a Double-A system (even an A/A-/Fail system), since students are not rewarded with an A- for doing minimal work. There is an incentive do continue doing work. There is a 20 point interval for each grade, which provides a cushion for students who, for a number of reasons, cannot perform as well in their classes from home as they could otherwise. This system also takes into account students' grades received so far in the semester, so the previous grades are not put to waste. 
\par We could also have Pass be a 4.0 and Low Pass be a 3.0, which allows students to increase their GPA in a fair manner (as opposed to the Double-A system), and which still makes GPA a useful metric. Even if there is no GPA increase tied to this system, employers, grad schools, etc. would still be able to see Pass vs Low Pass in student transcripts, which would be more meaningful than simply seeing a Pass or a Fail.

\newpage
\section{Supplemental Proposals}

\begin{enumerate}
    \item As an alternative for written exams, tell instructors to replace them with assignments or online exams that are not timed and are open-note. It is unethical to force students to install Honorlock or other remote proctoring and cheat detection systems, since they are invasive. Also, students will find a way to break these systems and cheat anyway. We should not force students to use invasive software whenever possible.
    \item Final exams should be replaced with final projects, if possible. Having students complete projects would not only eliminate the need for them to stress out by studying and memorizing facts, but it would also force them to apply their knowledge and skills to more practical scenarios. This would obviously not work for theoretical classes such as MATH 410 (Advanced Calc I) but it would be a very practical alternative for many classes in the humanities and sciences.
\end{enumerate}

\newpage
\section{Conclusion}
This concludes my arguments. Thank you for reading my document. I know the solution I propose is not perfect, but I think it is a better alternative compared to the ideas that have been proposed already. Please take some time to think about my ideas. You can give me your opinions at \textbf{cmurthy [AT] umd [DOT] edu}.

\end{document}